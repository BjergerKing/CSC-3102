\documentclass[12pt]{article}
\usepackage[margin=1in]{geometry}
\usepackage{amsmath}
\usepackage{amssymb}
\usepackage{fancyhdr}
\usepackage{enumitem}

\pagestyle{fancy}
\fancyhf{}
\rhead{James Singleton}
\lhead{CSC 3102: Homework 2}
\rfoot{\thepage}

\begin{document}

%Question 1////////////////////////////////////////////////////////
\begin{flushleft}
1. Show that $\log_{2}(n+1)$ is $O(\log n)$.
\end{flushleft}
\begin{align*}
\log_{2}(n+1)&\leq c\cdot \log_{2}n &&\text{for } n \geq n_{0}\\
\log_{2}(n+1) &\leq 3\log_{2}n &&\text{let }c=3\\
\log_{2}(3) &\leq 3\log_{2}2 &&\text{let }n_{0}=2\\
\log_{2}(3) &\leq 3\\
2^{\log_{2}(3)} &\leq 2^3\\
3 &\leq 8
\end{align*}
\begin{center}
$\therefore \log_{2}(n+1) \text{ is } O(\log n)$
\end{center}
\newpage

%Question 2//////////////////////////////////////////////////////
\begin{flushleft}
2. Consider the following program.\\[.2in]
$\quad for(i=0;i<n/2;i++)$\\
$\qquad for(j=0;j<n/2;j++)$\\
$\quad \qquad \text{print}(j)$\\[.2in]
\end{flushleft}
%Part A
\begin{enumerate}[label=A.]
\item Assume the print statement is $O(b)$. Write an expression $f(n)$ that gives the run time of this code in the form of summations ($\sum$).
\end{enumerate}
\begin{align*}
f(n)=\sum_{i=0}^{n/2} \sum_{j=0}^{n/2} b\\
\end{align*}
%Part B
\begin{enumerate}[label=B.]
\item Simplify your answer $f(n)$ from Part A as much as possible. Show your work.
\end{enumerate}
\begin{align*}
&f(n)=\sum_{i=0}^{n/2} \sum_{j=0}^{n/2} b\\
&f(n)=\sum_{i=0}^{n/2} b(n/2)\\
&f(n)=b(n/2)(n/2)\\
&f(n)=\frac{b}{4}(n^2)\\
\end{align*}
%Part C
\begin{enumerate}[label=C.]
\item Justify the Big-$O$ order of your answer $f(n)$ from Part B by finding $c$ and $n_{0}$ such that the definition of $O$ holds.
\end{enumerate}
\begin{align*}
\frac{bn^2}{4} &\leq c\cdot n^2 &&\text{for } n\geq n_{0}\\
\frac{n^2}{4} &\leq n^2&&\text{let } b=1,c=1\\
\frac{1}{4} &\leq 1
\end{align*}
\begin{center}
$\therefore f(n) \text{ is } O(n^2)$
\end{center}
\newpage

%Question 3////////////////////////////////////////////////////////
\begin{flushleft}
3. Consider the following very slightly different program.\\[.2in]
$\quad for(i=0;i<n/2;i++)$\\
$\qquad for(j=0;j<i/2;j++)$\\
$\quad \qquad \text{print}(j)$\\[.2in]
\end{flushleft}
%Part A
\begin{enumerate}[label=A.]
\item Assume the print statement is $O(b)$. Write an expression $f(n)$ that gives the run time of this code in the form of summations ($\sum$).
\end{enumerate}
\begin{align*}
f(n)&=\sum_{i=0}^{n/2}\sum_{j=0}^{i/2}b
\end{align*}
%Part B
\begin{enumerate}[label=B.]
\item Simplify your answer $f(n)$ from Part A as much as possible. Show your work.
\end{enumerate}
\begin{align*}
f(n)&=\sum_{i=0}^{n/2}b\cdot\frac{i}{2}\\
&=\frac{b}{2}\sum_{i=0}^{n/2}i\\
&=\frac{b}{2}\Bigg(\frac{n/2(n/2+1)}{2}\Bigg)\\
&=\frac{b}{4}\Bigg(\frac{n^2}{4}+\frac{n}{2}\Bigg)\\
&=\frac{b}{4}\Bigg(\frac{n^2}{4}+\frac{2n}{4}\Bigg)\\
&=\frac{b}{16}\big(n^2+2n\big)\\
\end{align*}
%Part C
\begin{enumerate}[label=C.]
\item Justify the Big-$O$ order of your answer $f(n)$ from Part B by finding $c$ and $n_{0}$ such that the definition of $O$ holds.
\end{enumerate}
\begin{align*}
\frac{b}{16}\big(n^2+2n\big)&\leq c\cdot n^2 &&\text{for } n\geq n_{0}\\
\frac{1}{16}\big(n^2+2n\big)&\leq n^2 &&\text{let } b=1,c=1\\
n^2+2n&\leq 16n^2\\
1+2&\leq 16 &&\text{let } n_{0}=1\\
3&\leq 16
\end{align*}
\begin{center}
$\therefore f(n) \text{ is } O(n^2)$
\end{center}
\newpage

%Question 4////////////////////////////////////////////////////////
\begin{flushleft}
4. Consider the following recursive definition of the binary search. It seeks a value $x$ in array $A$ and would be invoked as search($A$, 1, $n$, $x$). Assume $A$ is indexed beginning with 1.\\[.2in]
$\quad \text{search}(A, f, n, x)$\\
$\qquad \text{if}(n=1)$\\
$\quad \qquad \text{return } A[n]$\\
$\qquad \text{else}$\\
$\quad \qquad m=n/2$\\
$\quad \qquad \text{if}(x<A[f+m])$\\
$\qquad \qquad \text{return search}(A, f, m, x)$\\
$\quad \qquad \text{else}$\\
$\qquad \qquad \text{return search}(A, f+m, n-m, x)$\\
\end{flushleft}
%Part A
\begin{enumerate}[label=A.]
\item Show an example execution given the input\\ $A=[11,12,13,14,15,16,17,18,19,20,21,22,23,24,25,26]$ with $x=13$.\\
Specifically, list all calls to the search function in the order in which they occur, with arguments.
\end{enumerate}
\begin{center}
\textit{in order beginning with the initial call}\\
search($A$, 1, 16, 13)\\
search($A$, 1, 8, 13)\\
search($A$, 1, 4, 13)\\
search($A$, 3, 2, 13)\\
search($A$, 3, 1, 13)\\
\end{center}
%Part B
\begin{enumerate}[label=B.]
\item Write an expression $f(n)$ that gives the run time of this code in the form of a recursion. Assume the cost of calculating $m$ is $O(b)$. For simplicity, you may assume the cost of the conditionals, returns, additions, and subtractions are $O(1)$.
\end{enumerate}
\begin{align*}
f(n)=
\begin{cases}
1 & \text{if } n=1\\
b+1+f\big(\frac{n}{2}\big) & \text{otherwise}
\end{cases}
\end{align*}
\newpage
%Part C
\begin{flushleft}
4. \textit{(continued from previous page)}
\end{flushleft}
\begin{enumerate}[label=C.]
\item Simplify your answer $f(n)$ from Part B as much as possible, eliminating the recursion. Show your work.
\end{enumerate}
\begin{align*}
f(n)&=b+1+f\Big(\frac{n}{2}\Big) &&i=1\\
&=b+b+2+f\Big(\frac{n}{4}\Big)  &&i=2\\
&=2b+2+f\Big(\frac{n}{4}\Big)\\
&=2b+b+3+f\Big(\frac{n}{8}\Big)  &&i=3\\
&=3b+3+f\Big(\frac{n}{8}\Big)\\
&=ib+i+f\Big(\frac{n}{2^i}\Big) &&\frac{n}{2^i}=1 \text{ when } i=\log_{2}n\\
&=b\log_{2}n+\log_{2}n+1\\
&=2b\log_{2}n + 1
\end{align*}
%Part D
\begin{enumerate}[label=D.]
\item Justify the Big-$O$ order of your answer $f(n)$ from Part C by finding $c$ and $n_{0}$ such that the definition of $O$ holds.
\end{enumerate}
\begin{align*}
2b\log_{2}n+1 &\leq c \log_{2}n &&\text{for } n\geq n_{0}\\
\log_{2}n+1 &\leq 2 \log_{2}n &&\text{let } b=1,c=4\\
2^{\log_{2}n+1} &\leq 2^{2 \log_{2}n}\\
n\cdot 2 &\leq n^2 &&\text{let } n_{0}=2\\
4 &\leq 4
\end{align*}
\begin{center}
$\therefore f(n) \text{ is } O(\log n)$
\end{center}

\newpage

%Question 5////////////////////////////////////////////////////////
\begin{flushleft}
5. Simplify the following recursion to a non-recursive form.
\end{flushleft}
\begin{center}
$$f(n)=
\begin{cases}
a & \text{if } n=1\\
bn+3f\big(\frac{n}{3}\big) & \text{otherwise}
\end{cases}
$$
\end{center}
\begin{align*}
f(n)&=bn+3f\Big(\frac{n}{3}\Big) && i=1\\
&=bn+3\bigg(\frac{bn}{3}+3f\Big(\frac{n}{9}\Big)\bigg) &&i=2\\
&=bn+bn+9f\Big(\frac{n}{9}\Big)\\
&=2bn+9f\Big(\frac{n}{9}\Big)\\
&=2bn+9\bigg(\frac{bn}{9}+3f\Big(\frac{n}{27}\Big)\bigg) &&i=3\\
&=2bn+bn+27f\Big(\frac{n}{27}\Big)\\
&=3bn+27f\Big(\frac{n}{27}\Big)\\
&=ibn+3^if\Big(\frac{n}{3^i}\Big) &&\frac{n}{3^i}=1 \text{ when } i=\log_{3}n\\
&=(\log_{3}n)bn+3^{\log_{3}n}a\\
&=bn(\log_{3}n)+an\\
\end{align*}

\end{document}
